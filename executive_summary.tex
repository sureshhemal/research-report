\newpage
\section*{\centering Executive Summary}
\addcontentsline{toc}{section}{Excecutive Summary}


This study aims to assess the performance evaluation of a \ac{WWTP} located at Moratuwa/Ratmalana during a selected period in 2023. Data were collected based on physiochemical parameters such as pH and temperature, as well as chemical parameters, including \ac{BOD}, \ac{COD}, \ac{TSS}, \ac{OG}, orthophosphate (Dissolved Phosphate), and fluoride from the inlet and outlet. Under sludge management, sludge samples were collected from the sludge storage tank, return water and dewatered sludge samples from the belt press, and dried sludge samples from the drying bed to test for \ac{TSS}, moisture content, and dry solid content.

This study was conducted to accomplish several objectives, such as evaluating the effluent quality based on the standards issued by the \ac{CEA}, Sri Lanka, evaluating the efficiencies of removing contaminants through the biological treatment process, evaluating the belt press efficiency, and analyzing the sludge drying process and moisture removal percentage.

The performance evaluation presents the following removal efficiencies: 93.55\% of \ac{BOD}, 86.95\% of \ac{COD}, 95.28\% of \ac{TSS}, 97.04\% of \ac{OG}, 79.08\% of orthophosphate, and 68.04\% of fluoride.  This study also investigated the efficiency of the belt press dewatering process under two criteria and the moisture removal percentages during the drying period.  It was revealed that the study findings can be used to identify and fix operating and maintenance issues and plan future plant expansions to accommodate higher hydraulic and organic loads.
