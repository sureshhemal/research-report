\newpage
\section{DISCUSSION}

\subsection{Wastewater Treatment}
During this period, the first consideration was the flow of the sampling dates, since the analysis of wastewater flow data offers useful insights into the operational dynamics and performance of the treatment facility. Based on the flow during this period, there were several fluctuations due to the weather changes. Despite the varied flow, the amount of gully unloaded during the day did not affect it heavily. Within the study period, there were both dry and wet periods, and the precipitation might have affected the daily flow of the treatment plant.

Temperature is an important factor in wastewater treatment operations since it impacts microbial treatment effectiveness. Changes in influent temperature might impact the efficiency of biological treatment due to the influence of microbial function. The temperature of influent wastewater is influenced by factors like ambient temperature, sewage system architecture, and weather conditions like rain \cite{Brehar2019}. Also, releasing effluent with higher temperatures into waterbodies could affect the ecosystem around the area. Therefore, the effluent temperature should be in compliance with regulatory standards. During the period, the temperature requirement of effluent, as outlined in the Gazette (2022), was within the allowance limits \cite{CEA2022}.

Similar to temperature, pH also affects the efficiency of the treatment by influencing the growth and activity of microorganisms. Releasing effluent with very high or low pH values can harm the environment by negatively affecting water bodies and aquatic habitats. The tolerance limit of pH for a short sea outfall is outlined in the range of 5.5-9.0 (Minister of Environment, 2022), and the effluent pH range did not exceed the tolerance limit during the study period. 

When considering the \ac{BOD5} and \ac{COD} concentrations, the \Cref{fig:bod_graph} \& \Cref{fig:COD_graph} show a lower concentration of the effluent in both parameters. The tolerance limit of both \ac{BOD5} and \ac{COD} parameters did not exceed the limit during the study period, and several samples showed influent \ac{BOD5} and \ac{COD} values less than the tolerance limit for short sea outfall outlined as 75 mg/l and 400 mg/l, respectively \cite{CEA2022}. The acceptable concentrations of \ac{BOD5} and \ac{COD} for the treatment plant are identified as 355 mg/l and \numprint{1057} mg/l, respectively. There is a large variation between influent \ac{BOD5} and \ac{COD} concentrations and each acceptable concentration of \ac{BOD5} and \ac{COD} in more samples. 

The average removal efficiencies of \ac{BOD} and \ac{COD} are 93.55\% and 86.95\%, respectively, at a satisfactory level. However, compared to the performance evaluation done in 2016, there is a slight decrease in both \ac{BOD} and \ac{COD} removal efficiencies.

Higher influent \ac{TSS} values indicate the presence of high suspended solids in the incoming wastewater, which might complicate the treatment process and affect the treatment efficiency. Effluent \ac{TSS} levels indicate the treatment plant's capacity to eliminate suspended particles using sedimentation, filtration, and biological treatment methods.

Changes in effluent \ac{TSS} levels suggest fluctuations in treatment efficiency, equipment issues, or the necessity for process improvement. During the period, the \ac{TSS} variation between influent and effluent has shown higher values. In terms of the effluent quality, \ac{TSS} concentration did not exceed the tolerance limit outlined as 50 mg/l \cite{CEA2022} and the efficiencies during the period are at the satisfactory level indicated as 95.28\%. In comparison to the evaluation done in 2016 \cite{Danushika2016}, there is a small decrease in the \ac{TSS} removal efficiency in 2023.  Considering the acceptable influent \ac{TSS} concentration for the treatment plant, which is outlined as 458 mg/l, only 7 samples have shown higher concentrations. Therefore, the capacity is adequate to remove more \ac{TSS} amount.

\ac{OG} removal efficiency has varied during the period, as shown by changes in effluent concentrations. The removal efficiency of \ac{OG} was measured as 97.04\%. It indicates a satisfactory level of removing \ac{OG}. Due to the limited biodegradability of \ac{OG}, the discharge of most compounds in this category into the environment through wastewater could affect the biosphere \cite{Wahi2013}. A minimal amount of oil can impact marine life by reducing light penetration and oxygen transmission between air and water \cite{Roques2011}. The effluent quality is also at a satisfactory level since it does not exceed the tolerance limit outlined as 12 mg/l \cite{CEA2022}. 

With respect to the orthophosphate concentration, also known as dissolved phosphate concentration, and the fluoride concentration, the effluent concentration does not exceed the limits outlined as 5 mg/l and 2 mg/l, respectively. However, both influent and effluent concentrations of fluoride do not exceed the tolerance limit for short sea outfall outlined \cite{CEA2022}. Therefore, the effluent quality in both parameters is at a satisfactory level. The removal efficiencies of these two parameters were calculated as 79.08\% and 68.04\%, respectively.

The \ac{BOD5}/\ac{COD} ratio for untreated municipal wastewater usually ranges between 0.3 and 0.8. If the ratio equals or exceeds 0.5, the waste is considered as easily treatable using biological methods. If the ratio is less than 0.3, it is possible that the waste contains harmful substances or adapted bacteria that are required for degradation. For wastewater with very poor biodegradability, a treatment method is necessary to decrease the chemically oxidized organic content until the ratio falls within the 'biodegradable' range. The ratio is limited to 0.1-0.3 for the treated sewage \cite{AbdulRazzak2013}. Considering the influent biodegradability for this study period, many samples showed a moderate biodegradability range. Therefore, this wastewater is suitable for the biological treatment process. According to the effluent, the biodegradability range was between 0.1 and 0.3, which can be outlined within the biodegradability limit for treated water. There were some low values visible for influent biodegradability, possibly due to the industrial wastewater. Industrial wastewater has a low \ac{BOD5}/\ac{COD} ratio due to the high composition of \ac{COD} concentration. Therefore, it can reduce the influent biodegradability ratio. 

The overall efficiency of the treatment plant is at a satisfactory level, and fluctuations can be identified for the study period considered. The average removal efficiencies of \ac{BOD}, \ac{TSS}, and \ac{OG} are at higher levels of more than 90\%. However, the removal efficiencies of orthophosphate and fluoride are less than 80\%. 

According to the results, there is a possibility of treating the wastewater with higher effectiveness. Therefore, increasing domestic supplies of the treatment plant improves the effectiveness of the treatment process, and hence, the removal efficiencies and effluent quality would reach satisfactory levels.

Based on the correlation coefficient, p-value, F value, and R squared it can be concluded that a statistically significant positive relationship exists between the variables. A comparison of the calculated values using the regression equation and the actual tested values showed some variances. The variations may be due to the unloading of the gully bowsers with a higher concentration of \ac{TSS} and the influence of higher precipitation levels on weather conditions. However, this correlation can be utilized for estimating the range of the \ac{BOD5} level in the inlet in a short time period.

\newpage
\subsection{Sludge Management}
Sludge dewatering is an important step in sludge management, whereas drying sludge requires a significant amount of energy. Dewatering, the process of eliminating water without evaporation, is essential before heat drying is carried out. Dewatering wet sludge is crucial for decreasing sludge volume and transportation expenses, as well as enhancing the energy content of the sludge \cite{Chen2002}. The efficiency of a belt filter press is primarily assessed based on three criteria: the dry solid content of the dewatered sludge, the percentage of solids recovered, and the lateral sludge migration on the belt \cite{Olivier2005}. 

This study was based on the dry solid content of the dewatered sludge and the percentage of solids recovered. Typically, a dewatered sludge solids content of 15\% to 25\% is obtained when the feed solids content varies from 2\% to 5\% \cite{Mamais2009}. According to the dewatered sludge solid content, the dry solid content varies between 10\% and 14\%.  The solid recovered percentage during the period was 99\%. Therefore, the efficiency of the belt press under the percentage of solids recovered criteria is at a satisfactory level. However, the dewatered sludge dry content is less than the expected value. This can be due to the various sludge properties of the different sludges. The highest-impact enhancements in sludge dewatering were achieved by reducing the belt speed. At low speed, the dewatered sludge had a higher dry solid content, better filtrate quality, and less lateral sludge migration \cite{Olivier2005}. Since the belt press operates at a minimum belt speed in the treatment plant, the belt speed cannot be changed to increase dry solids. Hence, it is recommended to increase the initial dry solid content to increase the dry solid content of the dewatered sludge.

Drying is always a crucial step after mechanical dewatering. It can lower the water content to less than 5\% dry solids, reducing waste bulk and volume and eventually reducing costs for storage, processing, and transport. Drying sludge enhances the low calorific value, converting the product into a suitable fuel source. The product becomes pathogen-free and stabilized through treatment at high temperatures \cite{Leonard2008, Bennamoun2012}. Currently, the primary methods of sludge disposal are agriculture, incineration, and landfill. The use of solar energy is an effective strategy for reducing the expenses associated with the drying process. Understanding the drying process is crucial to effectively managing and controlling it. Unlike drying under constant conditions, the drying kinetics during solar drying exhibit fluctuations due to the continuously changing operating parameters \cite{Bennamoun2012}.

Considering the moisture content at different sample points, there is a huge fluctuation based on the sampling points due to the uneven distribution, and the average moisture content of the dried sludge varies between 35\% and 70\%. There are specific moisture contents for different disposal methods. In the first round, the moisture content of dried sludge was at a higher value, around 68\% due to the breakdown of the rotary. During the study period, the efficiencies also fluctuated due to climate change. Also, the moisture content of the sludge drying bed layers varied due to problems related to the mixing process. To avoid these issues, it is recommended to maintain the rotary properly to avoid frequent breakdowns and to add extra parts made out of rubber to increase the length of the cogs of the rotary to mix well between layers. The sludge mixing is limited to a single place when a rotary is used. Therefore, a higher fluctuation was identified between different sampling points. To avoid this issue, the use of an external source to mix the sludge by collecting and thinning the sludge several times during the drying time is essential. 