\newpage
\section{CONCLUSION}


Based on this performance evaluation of the wastewater treatment plant at Moratuwa/ Ratmalana, the removal efficiency of \ac{BOD}, \ac{COD}, \ac{TSS}, \ac{OG}, Ortho Phosphate and Fluoride are measured as 93.55\%, 86.95\%, 95.28\%, 97.04\%, 79.08\%, and 68.04\%. The effluent qualities fulfilled the standards issued by \ac{CEA}, Sri Lanka for releasing treated wastewater into water bodies. The removal efficiencies of Ortho-phosphate and Fluoride showed lower values than other parameters, but the effluent quality did not exceed the tolerance limits. The plant promises to treat the wastewater with high effectiveness and therefore, it is recommended to increase the domestic supplies connected to the \ac{WWTP}.

The belt press efficiency was carried under two criteria such as the percentage of solid recovered and the dry solid content of the dewatered sludge. The solid recovery percentage was at a satisfactory level but the solid content of the dewatered sludge was a little lower than the expected value.  Hence, it is recommended to increase the dry solid content by increasing the initial dry solid content. 

Considering the sludge moisture removal percentage, it is not easy to identify the specific moisture content because of not having a selected disposal method. For the enhancement of the removal percentage of the sludge, there are several recommendations such as proper maintenance of the rotary, adding extra cogs made of rubber to increase the length of the cogs of the rotary, and using the external source to mix the sludge in several times.