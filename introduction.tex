\newpage
\pagenumbering{arabic}
\section{INTRODUCTION}
Water is a fundamental requirement for all living organisms and is one of the most important resources on the earth \cite{Smarzewska2021}. The rapid population growth has impacted the resource utilization and lead to water pollution. Wastewater refers to liquids and solids from residential, commercial or agricultural operations, as well as other water utilized by humans, which is of lower quality and then released into a sewer system \cite{Prasad2020}.

Domestic wastewater can be categorized into two types, and they are greywater and blackwater. Greywater is the wastewater generated from bathtubs, laundry machines, hand basins, showers, and kitchen sinks, excluding any input from toilets. Approximately, 75\% of the total household sewage is composed of greywater \cite{Eriksson2002}. Blackwater originates from toilets and contains water, urine, excrement, and toilet paper materials \cite{Paulo2013}. The contaminants in wastewater can vary based on the industrial, agricultural, and municipal endeavours that release it. The pollutants are classified into three categories: inorganic, organic, and biological \cite{Sangeetha2023}.

This composition of the wastewater can result in adverse impacts on the environment, aquatic and wildlife populations and could potentially contaminate drinking water sources \cite{Smarzewska2021}. Organic matter can lead to oxygen depletion in rivers, lakes, and streams. Eventually, the biological decomposition of organic matter may lead to fish mortality and unpleasant smells. Also, wastewater includes nutrients that might enhance the growth of aquatic plants and may contain toxic substances that could be carcinogenic or mutagenic \cite{Prasad2020}. Therefore, water treatment is essential before discharge to the water bodies.

Wastewater treatment is a procedure that involves the partial removal and degradation of solids in wastewater, transforming highly complex organic compounds into mineral or relatively stable organic solids \cite{Sonune2004}. There are several Wastewater Treatment Plants \ac{WWTP} that are operated under the \ac{NWSDB} in Sri Lanka, located in Moratuwa/ Ratmalana, Kandy, Kurunegala, Jaela/ Ekala, Seethawake, Katharagama etc. These plants use different types of treatment processes, such as aerated lagoons, activated sludge treatment, and oxidation ditches. The activated sludge treatment process is a common technique used to treat wastewater, including municipal sewage and industrial wastewater \cite{AGUILAR2013} \cite{Chukwu2018}. Treated water from these \ac{WWTP}s are discharged by short sea outfall (600 m away from sea shore) and to the surface water in Sri Lanka.

The “Sludge” is identified as a solid by-product of biological wastewater treatment that contains harmful substances such as heavy metals, pathogens, and organic contaminants \cite{Wu2020}, and the wet sludge can contain up to 98\% of water content  \cite{Chan2016} \cite{SyedHassan2017}. Sewage sludge production has increased globally due to rising population, industrialization and urbanization and characteristics of the sludge vary with the source of the wastewater. Dewatering and drying are important procedures to reduce the volume for easy transportation and further sludge treatment \cite{Chang2023}. With reference to the \Cref{fig:Sludge_water}, the water content can be classified as (i) free (or Bulk), (ii) vicinal (or surface), (iii) interstitial, and (iv) chemically bound (or hydration) based on the interaction with sludge solid particles \cite{SyedHassan2017} \cite{Qi2020} \cite{Vaxelaire2004} \cite{Novak2006}. Free water, which can be eliminated through drainage, mechanical dewatering or thickening, refers to water that is not connected to or affected by the suspended solid particles. It is the most readily removable water from the wet sludge. Interstitial water is confined within the fissures and interstitial spaces of the flocs and organisms.

\begin{figure}[H]
\centering
\fbox{\includegraphics[width=.75\textwidth]{moisture distribution.jpg}}
\caption{Classification of water in sewage sludge}\cite{Qi2020}
\label{fig:Sludge_water}
\end{figure}

 Water can be transitioned into free water when the flocs and microbial cells are eliminated. Applying mechanical energy can remove a significant amount of the water by pressing it out. Many dewatering techniques can remove both interstitial water and bulk water from the sludge. Dewatered sewage sludge typically includes 73–84\% moisture \cite{Chan2016}.
 
 Sewage sludge management is financially and environmentally complex due to social challenges, high treatment expenses, health risks, and limited sustainable disposal methods \cite{Fuerhacker2011}. Some disposal methods for sludge are as fertilizers, some as fuel, and some as a raw material for building materials \cite{Bratina2016}.
 
 Evaluating the performance of a treatment plant involves measuring its efficiency using known indicators such as the removal of pollutants such as Biochemical Oxygen Demand \ac{BOD}, Chemical Oxygen Demand \ac{COD}, and suspended Solid contaminants. Operational difficulties can be detected through these evaluations and may be fixed to ensure that the plant functions properly \cite{Khan2018}. The performance efficiency of a treatment plant relies not only on its correct design and construction but also on proper operation and maintenance \cite{Kumar2010}.

 
 Therefore, this study was designed to evaluate the performance of \ac{WWTP} and sludge management.


 The objectives are to; 
 
 (1) evaluate the effluent quality based on the standards. 
 
 (2) analysis of the removal efficiencies of the contaminants through the biological treatment.
 
 (3) analysis of the sludge dewatering and drying process.